%%%%%%%%%%%%%%%%%%%%%%%%%%%%%%%%%%%%%%%%%%%%%%%%%%%%%%%%%%%%%%%%%%%%%%%%%%%%%%%
%
% witseiepaper-2005.tex
%
%                       Ken Nixon (12 October 2005)
%
%                       Sample Paper for ELEN417/455 2005
%
%%%%%%%%%%%%%%%%%%%%%%%%%%%%%%%%%%%%%%%%%%%%%%%%%%%%%%%%%%%%%%%%%%%%%%%%%%%%%%%%

\documentclass[10pt,twocolumn]{witseiepaper}
%
% All KJN's macros and goodies (some shameless borrowing from SPL)
\usepackage{KJN}
\usepackage[super]{nth}
\usepackage{subcaption}
\usepackage{listings}
\usepackage{amsmath}
\usepackage{epstopdf}
\usepackage{xcolor}
\usepackage{textcomp}
\usepackage{listings}
\usepackage{alltt}
%\usepackage{matlab-prettifier}
\usepackage{graphicx}
\usepackage{changes}
\usepackage{makecell}
\usepackage{verbatim}
\usepackage{algorithm,algpseudocode}
\usepackage{balance}
\usepackage{pdfpages}
\usepackage{ragged2e}
\usepackage{color} %red, green, blue, yellow, cyan, magenta, black, white
\definecolor{mygreen}{RGB}{28,172,0} % color values Red, Green, Blue
\definecolor{mylilas}{RGB}{170,55,241}
%\usepackage{flafter}

%
% PDF Info
%
\ifpdf
\pdfinfo{
/Title (INSTRUCTIONS AND STYLE GUIDELINES FOR THE PREPARATION OF FINAL YEAR LABORATORY PROJECT PAPERS : 2005 VERSION)
/Author (Ken J Nixon)
/CreationDate (D:200309251200)
/ModDate (D:200510121530)
/Subject (ELEN417/455 Paper Format, 2005)
/Keywords (ELEN417, ELEN455, paper, instructions, style guidelines, laboratory project)
}
\fi

%%%%%%%%%%%%%%%%%%%%%%%%%%%%%%%%%%%%%%%%%%%%%%%%%%%%%%%%%%%%%%%%%%%%%%%%%%%%%%%
\begin{document}
	
\begin{titlepage}
	
	\newcommand{\HRule}{\rule{\linewidth}{0.3mm}} % Defines a new command for the horizontal lines, change thickness here
	
	\center % Center everything on the page
	
	%----------------------------------------------------------------------------------------
	%	HEADING SECTIONS
	%----------------------------------------------------------------------------------------
		\includegraphics[width=0.3\textwidth]{EIE.png}\\[1cm] % Include a department/university logo - this will require the graphicx package
	
	%----------------------------------------------------------------------------------------
	\textsc{\LARGE University of the Witwatersrand } \\[0.1cm] % Name of your university/college
	\textsc{\LARGE School of Electrical and Information Engineering }\\[1cm] % Major heading such as course name
	\textsc{\Large ELEN4020: Data Intensive Computing}\\[1.5cm] % Minor heading such as course title
	
	%----------------------------------------------------------------------------------------
	%	TITLE SECTION
	%----------------------------------------------------------------------------------------
	
	\HRule \\[0.4cm]
	{ \huge \bfseries Laboratory Exercise 1} \\[0.4cm] % Title of your document
		\HRule \\[1.5cm]

%----------------------------------------------------------------------------------------
%	AUTHOR SECTION
%----------------------------------------------------------------------------------------
\textsc{\Large 	\emph{Authors:} } \\[0.1cm]	 


\begin{minipage}{0.4\textwidth}
	\begin{flushleft} \large
		%			\emph{Author:} \\
		Kayla-Jade Butkow \\ 714227 % Your name
	\end{flushleft}
\end{minipage}
~
\begin{minipage}{0.4\textwidth}
	\begin{flushright} \large
		%	\emph{Author:}\\
		Jared Ping \\
	\end{flushright}
\end{minipage}\\[1cm]

\begin{minipage}{0.4\textwidth}
	\begin{flushleft} \large
		%		\emph{Author:}\\
		Lara Timm \\ 704157
	\end{flushleft}
\end{minipage}
~
\begin{minipage}{0.4\textwidth}
	\begin{flushright} \large
		%		\emph{Author:} \\
		Matthew van Rooyen \\ 
	\end{flushright}
\end{minipage}\\[1cm]
		
\end{titlepage}


%%%%%%%%%%%%%%%%%%%%%%%%%%%%%%%%%%%%%%%%%%%%%%%%%%%%%%%%%%%%%%%%%%%%%%%%%%%%%%%
\thispagestyle{empty}
\pagestyle{plain}
\setcounter{page}{1}

\section{Introduction}


\section{PThread and OpenMP Libraries}

\subsection{PThread Library}
PThreads is a standardized model for dividing a program into parallel tasks \cite{pthreads}. PThreads was defined by the IEEE POSIX operating system interface standards \cite{pthreads}. The PThreads library specifies the interface to manage the actions required by threads \cite{pthreadVSopen}.

In order to use the PThreads library, the code must be written specifically for the library \cite{pthreadVSopen}. This involves the use of PThread specific functions and data structures. The implication of this is that once the library has been used, the application becomes threaded

\section{Conclusion}

\bibliographystyle{witseie}
\bibliography{dataLab1}

\end{document}

" vim: ts=4
" vim: tw=78
" vim: autoindent
" vim: shiftwidth=4
